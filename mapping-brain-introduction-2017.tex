\documentclass[12pt]{article}
\usepackage[margin=1in]{geometry}
\geometry{letterpaper}     
\usepackage[hyphens]{url}
\usepackage{fancyhdr}
\pagestyle{fancy}
\usepackage{enumitem}

\usepackage[parfill]{parskip}    % Activate to begin paragraphs with an empty line rather than an indent
\usepackage{graphicx}
\usepackage{amssymb}
\usepackage{epstopdf}
\usepackage{url}
\DeclareGraphicsRule{.tif}{png}{.png}{`convert #1 `dirname #1`/`basename #1 .tif`.png}

\title{Mapping the Brain:  An Introduction to Connectomics}
\date{}
\begin{document}
\vspace{-0.5cm}
\maketitle
\vspace{-2cm}
\begin{center}\textit{Draft Syllabus as of \today}\end{center}

\section*{Course Overview}
\begin{itemize}[noitemsep]
\item{\textbf{Course Number:}  600.221.13}
\item{\textbf{Credits:} 2 }
\item{\textbf{Class Meeting:}  Monday, Tuesday, Thursday, 2:45p-6p. Croft Hall	B32.}
\item{\textbf{Class Makeup:} There will be no class on Monday, January 16th because of MLK Day, and we will make up that class on January 18th.}
\item{\textbf{Office Hours:}  Upon Request}
\item{\textbf{Prerequisites:} None.  A background in neuroscience or computer programming is helpful but not required or assumed.}
\item{\textbf{Instructors:}

William Gray Roncal, Ph.D., JHU/APL Research Engineer.  240-338-8536, wgr@jhu.edu.\\
Gregory Kiar, M.S., JHU/Kavli Research Engineer. 443-554-6865, gkiar@jhu.edu.}
\item{\textbf{Course website:}  \url{http://github.com/connectomix}  }
\end{itemize}

\section*{Course Description} 
This course will introduce the emerging field of connectomics, and give students the opportunity to contribute directly to ongoing research efforts within the computer science department.  This field enables novel brain circuit analysis at the ultrastructure level (i.e., individual synapses and neurons) and promises insight into areas such as biofidelic algorithms and the validation of the cortical column hypothesis first proposed at JHU by Vernon Mountcastle in the 1960s.  We will begin by broadly surveying the field of brain mapping across  different scales, and more deeply examine research in ultrastructure electron microscopy reconstruction efforts.  Students will learn about scalable algorithms and approaches to extract graphs from large image volumes (on the order of 100 TB), and the importance of computer science in addressing modern neuroscience challenges.  

This year students will complete three small group projects: 

\begin{itemize}[noitemsep]
\item Billions for Big Brains:  design a large scale proposal for understanding \\ brain connectivity
\item MR Connectomics:  QUACK (Quality, Updating Algorithms, and \\ Creating Knowledge) project
\item EM Connectomics:  Proofreading the Brain at nanoscale resolution.
\end{itemize}
\section*{Grading Policy and Deliverables}

This course is intended to provide insights into a new and growing field, and strengthen your background as scientists, and so is offered pass/fail (S/U).  There are no exams, however students are expected to attend all classes unless prior arrangements have been made.  To receive a grade of Satisfactory, students should:

\begin{itemize}[noitemsep]
\item{Attend every class.  Students missing more than one class may not receive credit.}
\item{Participate.  Students are expected to be engaged in class, complete assignments individually and in groups and ask questions regularly.}
\item{Actively present in class.}
\item{Other assignments as given in class.}
\end{itemize}

This class will move quickly given the short time frame; the amount you learn is strongly correlated with the time and effort you contribute!  Students without any prior background in this field are welcome and will be able to succeed.
\vspace{-0.5cm}
\section*{Course Schedule}
Each class will consist of a lecture, paper presentations, workshop activity, and group project lab.  We will explore a different facet of connectomics each class:
\begin{itemize}[noitemsep]
\item{Class 1:  Processing a Connectome}
\item{Class 2:  Scientific Writing}
\item{Class 3:  Neuroscience, Graphs and Inference}
\item{Class 4:  Reproducible and Extensible Science}
\item{Class 5:  Connectomics in Humans}
\item{Class 6:  Grad School and Research}
\item{Class 7:  Data science}
\item{Class 8:  Nanoscale Synapses}
\item{Class 9:  Nanoscale Neurons}
\end{itemize}
\vspace{-0.5cm}
\section*{Honor Code}
For this course, collaboration is encouraged, and referencing outside resources is permitted, unless expressly noted.  The work you turn in must be original and your own - you are required to acknowledge significant help received, as well as sources consulted (i.e., citations).  
\vspace{-0.5cm}
\section*{Final Thoughts}
We are excited to teach you about the field of connectomics and support you however we can.  Please feel free to reach out (Slack is best) with questions or comments at any point.

Opportunities to continue research in the Spring semester may be available (at JHU/APL or JHU), for the motivated student.

\end{document}  